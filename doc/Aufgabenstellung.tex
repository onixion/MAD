\documentclass[10pt,a4paper]{report}
\usepackage[utf8]{inputenc}
\usepackage[german]{babel}
\usepackage{amsmath}
\usepackage{amsfonts}
\usepackage{amssymb}
\usepackage{graphicx}
\usepackage{fancyhdr}
\usepackage{pdflscape}
\usepackage{geometry}
\usepackage{pdfpages}

\geometry{a4paper, left=25mm, right=25mm, top=20mm, bottom=30mm}

\title{Diplomarbeit - Netzwerkmonitoring}
\author{Porcic Alin, Ranalter Daniel, Singh Manpreet, Stojanovic Marko}

\begin{document}
\maketitle
\chapter{Aufgabenstellung}
Die Aufgabenstellung dieser Diplomarbeit beinhaltet das Schreiben einer Software welche für Administratoren im Heimnetzwerk oder kleineren Firmen ausgelegt ist. Sie soll dem Administrator verschiedenste Informationen über die in dem Netzwerk angeschlossenen Geräte, seien es Server oder Clients, Hosts u.a., bieten. Diese Informationen sollen über Hilfsmittel wie Ping und den Abfragen von auswählbaren Diensten erfolgen. Zu solchen zählen beispielsweise HTTP oder DHCP und DNS. Auch ein genereller Portscan soll mit dabei sein um eine ganze Reihe an Ports zu überprüfen, um so bei unwichtigeren Diensten grob überprüfen zu können ob zumindest der Port noch offen oder auch geschlossen ist. Im Endeffekt soll für den User unseres Programmes dann ersichtlich sein welche Adressen, MAC und IP, die angeschlossenen Hosts haben, sowie welche Ports von diesen offen sind. Das Programm wird für den User entweder als graphische Oberfläche(in der weiteren Dokumentation wird hierauf als GUI für Graphical User Interface, referenziert) zur Verfügung stehen, oder aber als ein Command Line Interface (in der weiteren Dokumentation wird hierauf als CLI referenziert). Auch wird ein Client zur Verfügung stehen von welchen man aus von anderen Geräten auf das Programm zugreifen kann um kleine Änderungen vorzunehmen, welche unter Umständen als Antwort auf eine E-Mail welches die Software von sich aus in regelmäßigen Abständen oder Notsituationen verschicken soll, notwendig sein können. Das fertige Projekt wird für Linux sowie Windows unterstützt. Für die Realisierung wird die Sprache C\# mit dem .NET Framework für Windows und mono für Linux verwendet. Es wird als Endprodukt eine Installationsfile geben welche zwei Icons für entweder CLI oder GUI erstellt sowie die Datenbank welche zur Abspeicherung von Ereignissen dient und einer XML Datei welche die vom Administrator angewendeten Konfigurationen speichert. 
\end{document}